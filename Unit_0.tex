\section*{Словарь терминов и сокращений}
\noindent

\begin{tabular}{l l}
%{\bf БППП}  & --- Бюро планирования и подготовки производства.\\
{\bf г/к}, {\bf ГК}  & --- Гофрокартон.\\
{\bf ГП}  & --- Готовая продукция.\\
% {\bf ГОГП}  & --- группа по отпуску готовой продукции.\\
{\bf ЛГК}, {\bf ГА}, {\bf Г/А}  & --- Гофроагрегат.\\
{\bf ЛП}, {\bf ПЛ}, {\bf П/Л}  & --- Линия переработки, перерабатывающая линия.\\
%{\bf ОПП}  & --- Отдел подготовки производства.\\
% {\bf КИПиА}  & --- Контрольно--измерительные приборы и автоматика.\\

{\bf ИТР}  & --- Инженерно-технические работники.\\
%{\bf КБ}  & --- Конструкторское бюро.\\

%{\bf КРС}  & --- Контрольно--ревизионная служба.\\
%{\bf КК}  & --- Контрольная карта.\\
{\bf МС}  & --- Материальный склад.\\
%{\bf ОСЛ}  & --- Отдел сбыта и логистики.\\
{\bf ОП}  & --- Отдел продаж.\\
%{\bf ОГЭ}  & --- Отдел главного энергетика.\\
%{\bf ОЭАиК}  & --- Отдел экономического анализа и контроля.\\
%{\bf ОКС}  & --- Отдел капитального строительства.\\
%{\bf ОИТ}  & --- Отдел информационных технологий.\\
%{\bf ООТиОС}  & --- Отдел охраны труда и окружающей среды.\\
%{\bf ОУК}  & --- Отдел управления качеством.\\
% {\bf РТС}  & --- Ремонтно-техническая служба.\\
% {\bf ОГИ}  & --- Отдел главного инженера.\\
% {\bf ОГЭ}  & --- Отдел главного энергетика.\\
%{\bf ПДО}  & --- Планово-диспетчерский отдел.\\
%{\bf ППО}  & --- Планово-производственный отдел.\\
%{\bf ПСЦ}  & --- Паросиловой цех.\\
%{\bf ПТО}  & --- Производственно--технический отдел.\\
{\bf ИС}  & --- Информационная система.\\

% {\bf МАП}  & --- Менеджер активных продаж.\\
% {\bf МСЗ}  & --- Менеджер сопровождения заказов.\\

{\bf СГП}  & --- Склад готовой продукции.\\
{\bf ТМЦ}  & --- Товарно--материальные ценности.\\
%{\bf ТЭО}  & --- Технико-контрольный отдел.\\
%{\bf УВФ}  & --- Участок вырубных форм.\\
%{\bf ЦАК}  & --- Цех автокартона.\\
%{\bf ЦГК}  & --- Цех гофрокартона.\\
%{\bf ОПП}  & --- Отдел подготовки производства.\\
%{\bf ОТК}  & --- Отдел технического контроля.\\
%{\bf СМК}  & --- Система менеджмента качества.\\
{\bf ОУК}  & ---  Отдел Управления Качеством. \\
%{\bf РРК}  & --- Расчетно-раскройная карта.\\

\end{tabular}


\newpage

\chapter{Введение}

\section{Общие сведения}

Объектом обследования является \FIRMA, именуемое в дальнейшем ПРЕДПРИЯТИЕ.

Сокращенное название -- нет.

Предметом обследования являются производственные процессы \FIRMA. 


В своей деятельности ПРЕДПРИЯТИЕ обязано руководствоваться действующим законодательством РФ, Уставом, приказами и иными документами.

Основной целью деятельности ПРЕДПРИЯТИЯ является получение прибыли за счет выполнения заказов от сторонних организаций на изготовление упаковки из гофрированного картона и реализация товарного картона. 


ПРЕДПРИЯТИЕ выпускает следующие виды продукции.
\begin{enumerate}
%	\item гофрокартон: трехслойный профиля B,С, Z-картон;
    \item гофрокартон: трехслойный профиля B, С, Е и пятислойный профилей ЕВ, ЕС, ВС;
	\item упаковка из гофрокартона: гофроящики, в том числе тара сложной конфигурации (лотки, ложементы);
	\item комплектующие изделия из гофрокартона: решетки, прокладки и т.п.
\end{enumerate}

Юридический адрес ПРЕДПРИЯТИЯ --- \CURADDRESS

Адрес производства: \ADDRESS



\section{Основные показатели}

Полученные численные показатели на 20.01.2025.

\begin{itemize}
\item Количество сотрудников предприятия --- 306 человек;
%\begin{itemize}
%\item Цех  --- 1.

%\end{itemize}
\item Количество штатных единиц управленческого персонала --- XX;
\item Количество единиц производственного оборудования --- ХХ;
\item Количество производственных цехов --- 1;
\item Тип производства --- многономенклатурное;

\item Количество первичных документов:
\begin{itemize}
\item Среднее количество документов по закупке (сырье), ед/мес — XX;
\item Среднее количество документов по реализации, ед/мес — XXX;
\item Среднее количество заказов от заказчиков, ед/мес — XXX;
\item Среднее количество новых изделий в месяц, ед/мес — XXX;
%\item Среднее количество заказов поставщику, ед/мес — ???;
\item Среднее количество заказчиков в месяц — XX;
\item Количество контрагентов — XXX;
\item Количество видов номенклатуры изделий — XXX;
\item Количество номенклатурных позиций - XXX.


\end{itemize}

\end{itemize}












% \ifx \notincludehead\undefined
\normalsize
\end{document}
\fi