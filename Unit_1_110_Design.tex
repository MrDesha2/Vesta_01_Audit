%\newpage
\subsection{Подготовка производства}
\label{bp:Prepare}


Технолог-конструктор работает в системе 1С: УПП, используя обработку ''Рабочее место технолог-конструктор гофротара'' (рис. \ref{pic:II.2}).

Технолог-конструктор после получения информации от менеджера о создании новой технологической карты открывает ее (рис. \ref{pic:II.2.} - \ref{pic:II.4..}) и перепроверяет все заполненные поля менеджером.

Технолог-конструктор создает номенклатуру на оснастку в системе 1С: УПП. 

На вкладке ''Этапы утверждения'' (рис. \ref{pic:II.4...}) технолог-конструктор, используя команду ''Запустить в работу'', активирует ТК.

В системе 1С: УПП созданы справочники оснастки (рис. \ref{pic:II.5.3..}), доступно формирование отчета ''Реестр оснастки'' (рис. \ref{pic:II.5.3.}. 

Форма элемента справочника ''Оснастка'' приведена на рис. \ref{pic:II.5.3.}.

В системе 1С: УПП можно сформировать печатную форму ТК (рис. \ref{pic:II.5.4.}, \ref{pic:II.5.4}).

На ПРЕДПРИЯТИИ организован участок по изготовлению ротационной высечной оснастки. Разработкой чертежей для изготовления штанцформ занимается инженер-конструктор и техник-конструктор. Все чертежи хранятся в компьютерах конструкторов. 
На участке по изготовлению работают два мастера по ремонту и изготовлению оснастки, они же занимаются изготовление образцов ГП на плоттере. Реестр заданий на изготовления штанцформ ведется в MS Excel.

Разработкой макета печати занимаются сотрудники поставщика фотополимерных форм. Постановка задачи на разработку макета печати (рис. \ref{pic:II.8.7}) осуществляется в письменном виде по электронной почте. Реестр заданий на изготовление фотополимерных форм ведется MS Excel (рис. \ref{pic:II.8.7}). Все данные хранятся у технолога-конструктора.

Подготовкой краски в работу (ПРЕДПРИЯТИЕ оснащено станцией смещения красок) занимается специалист по подготовке производства. Также в его обязанности входит приемка ФПФ после выпуска ГП.




\clearpage

%\begin{figure}
%\begin{center}
% \includegraphics[height=0.4\textheight, keepaspectratio]{Pics/II.1.jpg}
%\end{center}
% \caption{Пример запроса менеджера технологу по подготовке производства}
% \label{pic:II.1}
%\end{figure}

\begin{figure}
\begin{center}
 \includegraphics[height=0.35\textheight, keepaspectratio]{Pics/II.2.jpg}
\end{center}
 \caption{Рабочий стол технолога по подготовке в 1С: УПП}
 \label{pic:II.2}
\end{figure}


\begin{figure}
\begin{center}
 \includegraphics[height=0.4\textheight, keepaspectratio]{Pics/II.2..jpg}
\end{center}
 \caption{Форма технологическая карта для проверки технологом по подготовке}
 \label{pic:II.2.}
\end{figure}

\begin{figure}
\begin{center}
 \includegraphics[height=0.4\textheight, keepaspectratio]{Pics/II.4.jpg}
\end{center}
 \caption{Вкладка ''Доставка, Упаковка, Оснастка'' в технологической карте}
 \label{pic:II.4.}
\end{figure}

\begin{figure}
\begin{center}
 \includegraphics[height=0.4\textheight, keepaspectratio]{Pics/II.4..jpg}
\end{center}
 \caption{Форма технологическая вкладка Дизайн}
 \label{pic:II.4..}
\end{figure}


\begin{figure}
\begin{center}
 \includegraphics[height=0.4\textheight, keepaspectratio]{Pics/II.4...jpg}
\end{center}
 \caption{Форма технологическая карта этапы утверждения статуса}
 \label{pic:II.4...}
\end{figure}

\begin{figure}
\begin{center}
 \includegraphics[height=0.4\textheight, keepaspectratio]{Pics/II.5.3...jpg}
\end{center}
 \caption{Форма справочник оснастки в 1С: УПП}
 \label{pic:II.5.3..}
\end{figure}

\begin{figure}
\begin{center}
 \includegraphics[height=0.4\textheight, keepaspectratio]{Pics/II.5.3.jpg}
\end{center}
 \caption{Отчет Реестр оснастки }
 \label{pic:II.5.3}
\end{figure}

\begin{figure}
\begin{center}
 \includegraphics[height=0.4\textheight, keepaspectratio]{Pics/II.5.3..jpg}
\end{center}
 \caption{Карточка оснастки}
 \label{pic:II.5.3.}
\end{figure}

\begin{figure}
\begin{center}
 \includegraphics[height=0.4\textheight, keepaspectratio]{Pics/II.5.4..jpg}
\end{center}
 \caption{Печатная форма технологической карты}
 \label{pic:II.5.4.}
\end{figure}

\begin{figure}
\begin{center}
 \includegraphics[height=0.4\textheight, keepaspectratio]{Pics/II.5.4.jpg}
\end{center}
 \caption{Печатная форма технологической карты, вид упаковки}
 \label{pic:II.5.4}
\end{figure}

\begin{figure}
\begin{center}
 \includegraphics[height=0.4\textheight, keepaspectratio]{Pics/II.8.7.jpg}
\end{center}
 \caption{Задание на разработку дизайна поставщику ФПФ}
 \label{pic:II.8.7}
\end{figure}

\begin{figure}
\begin{center}
 \includegraphics[height=0.3\textheight, keepaspectratio]{Pics/II.5.5.jpg}
\end{center}
 \caption{Реестр изготовления фотополимерных форм}
 \label{pic:II.5.5}
\end{figure}





% Клиент присылает требования на изготовления продукции в любом виде.
% Опросного листа не выявлено.
% Заказчики могут принести образец продукции, по которому инженер-конструктор производит замеры размеров или может выполнить расчет размеров короба по готовой продукции.
% % (рис. \ref{pic:a8}, \ref{pic:a9}). 

% Общее количество технологических карт на предприятии около 2500 шт. 
% Системы нумерации не выявлено.

% В производстве и в отделе подготовки производства технологические карты пользователи ищут по размерам изделия и по наименованию контрагента, что может привести к ошибке. 

% % Новых изделий ОПП заводят от 2 до 40 шт в день. 
% % Пример полного комплекта технологической карты приведен на форме \ref{pic:a5}.
% %Требования по новому изделию хранятся локально у МАП на компьютере или в почте. 

% %%
% %После согласования цены на основании калькуляции (рис. \ref{pic:d1}) МСЗ создает в таблице MS Excel техническое задание для дизайнера по шаблону. 
% %МСЗ в сети создает папку с новым номером изделия из 4 цифр. МСЗ создает новое техническое задание на разработку нового изделия, присваивает новый номер. 
% %Номер технологической карты --- это номер карты раскроя (рис. \ref{pic:d7}).
% %Номер ТК МСЗ переносит в калькуляцию вручную. Номер МСЗ указывает в форме спецификации (рис. \ref{pic:d6}). 
% %По наличию номера в печатной форме технологической карты МСЗ проверяет ее наличие.
% %МСЗ отправляет клиенту проект договора, спецификацию к договору и список технологических карт.
% %МЗС сохраняет предоставленные клиентом файлы в папку с технологическими картами. 

% %После согласования цены на основании калькуляции (рис. \ref{pic:d1}) МСЗ создает в таблице MS Excel техническое задание (рис. \ref{pic:d7}) на разработку технологической карты дизайнеру и сохраняет в папке с номером ТК. Номер присваивает МСЗ и создает папку с новым номером изделия из 4 цифр.
% %Номер технологической карты --- это номер карты раскроя (рис. \ref{pic:d7}).
% %Номер ТК МСЗ переносит в калькуляцию вручную. Номер МСЗ указывает в форме спецификации (рис. \ref{pic:d6}). 

% Менеджер при приемке нового изделия согласовывает возможность изготовления продукции по таблице (рис. \ref{pic:d28}). 
% Если есть вопросы по изготовлению, то  менеджер создает вопрос в техотдел. 
% % (рис. \ref{pic:d28}).

% При поступлении заявка на производство от отдела продаж (рис. \ref{pic:d2}) с пометкой «новая» в технологическом отделе сотрудники начинают разработку новой технологической карты (ТК).

% Разработкой ТК на четырехклапанный короб занимается инженер-конструктор технологического отдела. В шаблон ТК таблицы MS Excel инженер указывает  размеры четырехклапанного короба (рис. \ref{pic:a2}) и через формулы просчитывает необходимые параметры. Инженер создает ТК на четырехклапанный короб. Папка с созданными ТК на четырехклапанные короба хранится на сервере.
% % (рис. \ref{pic:a1}). 
% Папки разделены по клиентам. При отсутствии папки по клиенту менеджеры создают нового клиента и новую ТК. 

% Отдельно от ТК инженер ОПП разрабатывает схему упаковки и расчет укладки на поддон  в формате MS Excel (рис. \ref{pic:f3}). Схемы укладки хранятся отдельно в папке на сервере (рис. \ref{pic:d26_1}). 

% Также в технологическом отделе разрабатывают бирку на ГП и их хранят тоже отдельно в папке на сервере (рис. \ref{pic:a4}).

% При разработке нового дизайна менеджер присылает письмо на электронную почту художнику-конструктору с требованием о разработке дизайна нового изделия. 
% Художник-конструктор разрабатывает макет на основании данных присланных в письме, готовый макет высылает менеджерам для согласования с клиентом. 
% После согласования макета с клиентом художник-конструктор делает заказ клише в «Колор Стандарт Сервис» (см. процесс ''Учет технологической оснастки и краски'' \ref{bp:rigging}). Сканы подписанных клиентом макетов с дизайном хранятся на сервере в папках в сетевом доступе. % ф 3,4. 
% Художник-конструктор разрабатываем макеты в пакете CОRЕL DRAW. 

% Разработку конструкции сложной высечки выполняет инженер-конструктор. 
% Инженер-конструктор разрабатывает макет высечки в пакете AutoCAD (рис. \ref{pic:f10}). На предпритии в ходе обследования выявлена база наработок, чертежей, которые хранятся в сети. % ПК ф.12. 
% Макет конструкции инженер-конструктор согласовывает с клиентом через менеджера. 
% Согласованный чертеж инженер-конструктор отсылает в «РастрТехнологии» (см. процесс ''Учет технологической оснастки и краски'' \ref{bp:rigging}), где чертеж проверяют и наносят «шапку» % ф. 11, 
% и с присвоенными данными по чертежу высылают макет назад. Изготовитель штампа также высылает счет. %, после чего заказывается штамп. 
% Готовые чертежи хранят на сервере. 
% На сложную высечку ТК  хранятся отдельно.

% При изменении в ТК старая форма изымается из производства. 
% Технологичекие карты в печатном виде хранятся на участоке вырубных форм (рис. \ref{pic:f5}). % Ф. 6,7 ТК редко используемые хранятся отдельно. 

% Отдельно готовая технологическая карта с клиентом не согласовывается.



% %МСЗ сообщает дизайнеру о появлении технического задания.
% %В техническом задании МСЗ  указывает тип изделия и внутренние размеры. Заказчик может предоставить уже готовый чертеж, тогда МЗС  прикладывает дизайн от клиента. 
% %Созданное техническое задание на разработку изделия МСЗ  высылает дизайнеру на почту. 
% %Дизайнер разрабатывает ТК для основных изделий: гофроящик, лоток, решетка, гофролист.
% %Дизайнер разрабатывает ТК по запросу (рис. \ref{pic:d8}) в системе Corel Draw. Размеры изделия дизайнер указывает в программе Corel Draw по факту в миллиметрах. Дизайнер разрабатывает только простые изделия и макет печатной формы, выполняет расчет размеров заготовки вручную, в следствие чего существует большая вероятность появления ошибки. Дизайнер вручную рассчитывает размер поддона, рисует схему погрузки продукции в транспорт.

% %Дизайнер при необходимости разрабатывает дизайн печати на изделии. 
% %Дизайнер в программе Corel Draw рисует коробки, решетки для изготовления штанц-формы. Разработка конструкции сложной высечки не выполняется.
% %После разработки макета технологической карты дизайнер сообщает МСЗ о ее готовности по телефону или мессенджеру.






% %Менеджер в системе 1С CRM заполняет опросный лист (1). У инженера загорается задача в системе 1С CRM (ф1), при дальнейшей работе с этим опросным листом меняются автоматически статусы (ф2).

% %Для 4-х клапанного короба по опросному листу инженер смотрит применение коробки и присваивает ТУ, ТУ 068 – пищевой короб, ТУ 069- промышленный короб (ф3). Далее смотрят размеры, был ли короб ранее с такими размерами, если был, то ищут на сервере в таблице EXCEL (ф4) присвоенный ему номер, если не было ранее такого короба, то ему присваивают номер. Далее в EXCEL шаблоне создают ТК (ф5). В 1С CRM автоматически формируется артикул, который состоит из ТУ, размеров короба и варианта исполнения. (ф4а), (3,4).

% %Для проверки технологичности и возможности изготовления короба, применяют расчетный шаблон в EXCEL (ф5а), если проверку не прошло, но у инженера есть предположения, что данный короб могут изготовить, то отсылают на согласование на производство.

% %Иногда менеджер может принести образец от клиента и тогда инженер производит замеры и подбирают нужный короб. Для замеров решетки, могут принести пустые бутылки.
% %Если есть комплектующие, то в опросном листе 1С CRM будет указано к коробу есть решетка или решетка к коробу №….

% %На сложную высечку могут дать ссылку в каталоге FEFCO или принести чертеж от клиента. При обращении клиента через личный кабинет, и отсутствии чертежей, общение с клиентом может осуществлять инженер через личный кабинет, предлагая различные варианты. Могут прислать фото короба (ф6).

% %Сложную высечку чертят в программе AUTOCADE. (ф 6а). После разработки чертежа в 1С CRM в опросном листе, в графе ОПЗ присваивают номер чертежа (ф 7а). Переводят чертеж в PDF и прикрепляют в опросный лист для согласования. После согласования с клиентом, менеджер в программе 1С CRM запускает процесс заказа штампа. Служебная записка проходит согласование разных служб, в ней указывается доходность этого короба, для принятия решения руководителю. (ф 10). После согласования в AUTOCADE разрабатывают чертеж заготовки и раскладку на штампе (ф 11,12), присваивают им номера чертежей. Указывают направление гофры по сторонам (длинна или ширина) (ф 13). Далее присваивают номер ящика, если он был размерным, находят по номер по размеру. (ф 7,8). Все файлы прикрепляют в опросном листе в графе «файл» (ф 9, 14), затем отправляют на согласование в производство (ф 15). 

% %После согласования производства, в EXCELE таблице (ф 16) на сервере по линиям создают папку с номером ящика, куда вкладывают все чертежи для заказа штампа. Если штамп заказывают за счет клиента, то после оплаты заказывают штамп. Папку с чертежами отправляют в Растр технологии или Лазер Пак. (ф 16а). Изготовители присваивают номер заказа и перечерчивают чертеж штампа с указанием всех параметров (ф 17а), отправляют на согласование. Далее инженер согласовывает в программе 1С CRM (ф 17б) и отправляет назад согласованный чертеж штампа. Ожидают счет на оплату (ф 17в) и при его поступлении вкладывают в файл в папку снизу опросного листа, не для видимости клиента (ф 17г). 

% %О приходе штампа сообщает технолог из цеха или отслеживают сами. После прихода штампа заполняют таблицу EXCEL на сервере (ф 18а) и в папке паспорта (ф 19) создают паспорт на штамп (ф 20). После заполнения все чертежи отправляют в цех технологам, где они заполняют «эксплуатацию штампа».

% % При разработке макета печати используют файлы, вложенные в опросном листе (ф 21), реже подбирают логотипы из интернета или обрисовывают в CorelDRAW, масштабируют по коробу. Дизайн не разрабатывают. Рисунки от клиентов необходимы в векторном файле, если рисунок пришел не в соответствующем разрешении или более 3-х цветов, то отправляют на корректировку клиенту. 
 
% % Если все данные подходят, то из таблицы EXCEL на сервере (ф 22) берут порядковый номер и в программе 1С CRM занимают место с этим номером. 
 
% % Эскиза печати загружают в опросный лист в 1С CRM и отправляют на согласование в производство. (ф 23, 23а). После согласования производством, эскизы отправляют на согласование с клиентом, в личном кабинете или через менеджера.
 
% % Согласованные эскизы печати выкладывают на сервер в папку CDR (ф 24а), оттуда инженер берет эскизы и накладывает на ящик при создании ТК.
 
% % По согласованию с клиентом, на все короба ставят номерной штамп. На этом штампе меняют площадь и номер короба. (ф 24).
 
% % Затем менеджер создает заявку на заказ клише. Если в заявке стоит галочка «счет на клише» - это означает, что это клише необходимо заказать, если такой галочки нет, значит клише есть на производстве. (ф 25а).
 
% % После согласований, приходит задача дизайнеру на заказ клише в 1С CRM. Дизайнер готовит пакет чертежей и нужную документацию (ф 25, 25б). Если короб сложной высечки, то прикладывают чертёж заготовки. Создают заявку (ф 26) и весь пакет документов отправляют по электронной почте в РЕПРОПАРК. После обработки РЕПРОПАРК высылает макет на согласование и счет.
 
% % На ГЦ2 в основном заказывают полимеры, т.к. есть свой отдел по подготовки оснастки, где монтажисты наклеивают полимеры на фартук, на ГЦ1 такой отдел отсутствует и клише, адресованные на эту площадку, заказывают уже готовым комплектом.
 
% % Далее делается чертеж монтажа полимеров, который отправляют по электронной почте монтажистам. Все согласованные чертежи выкладывают на сервер (ф 26а).
 
% % После заказа оснастки, менеджер по работе с поставщиками отдела подготовки производства, отслеживает приход, о котором сообщат по электронной почте или в таблице.
 
% % Перед приходом оснастки, распечатывают счет (ф 27а), заносят в 1С участок вырубных форм и дублируют в EXCEL таблице на сервере (ф 27). При поступлении оснастки заносят приход в 1С участок вырубных форм, а в EXCEL таблицу вносят УПД.
 
% % Списание оснастки происходит по акту от производства (ф 28).
 
% % Один раз в пол года проводят инвентаризацию оснастки. В таблицу EXCEL вносят оснастку, не использованную за последние два года и отправляют менеджерам, для пометки о ее дальнейшей судьбе – отдать клиенту, оставить или утилизировать. 
 
% % При разработке ТК на 4-х клапанный короб, карта упаковки разрабатывается сразу в EXCEL. При сложной высечки, карта упаковки разрабатывается позже. После прикрепления чертежа в опросный лист, поступает задача инженеру на разработку упаковки. Инженер заполняет в 1С CRM габариты пачки (ф 29) и программа автоматически рассчитывает укладку на паллет, высоту паллеты и т.д. (ф 30). Номер карты упаковки формируется автоматически.
 
% % После инженер создает заявку на номенклатуру, она создается в 1С участок вырубных форм отделом планирования ГОЛОВАНОВО. В 1С участок вырубных форм карта упаковки загружается спецификацией. 
 
% % При планировании в отделе планирования видят поступление нового заказа без ТК, заходят в 1C CRM APM технологическая документация, по номенклатуре или артикулу находят необходимую ТК и подгружают ее (ф 31), также руками переносят в PC Topp. 
 
% % Общее количество ТК не известно, за последний год было разработано 5867 ТК.
% %Штампов более 1200, клише более 2000 штук.


% %Менеджер получает запрос на разработку нового изделия только по почте от клиента.

% %Заявка на изготовление нового изделия поступает от менеджера к дизайнеру в журнале MS Exсel.
% %При отсутствии на заявке артикула отдел учёта создаёт новое здание. Для существующих изделий присвоен артикул, который менеджер указывает в комментарии в заявке .

% %Менеджер формирует заявку (рис. \ref{pic:pic_d7}), высылает в отдел учета.  Отдел учёта создает номенклатуру в программе 1С: 7.7. Бухгалтерия по требованиям менеджера (рис. \ref{pic:pic_d7}). При отсутствии номенклатуры отдел учета создает новую номенклатуру в системе 1С: 7.7. Бухгалтерия и заявку в справочнике заявок. 

% %Отдел учета в системе 1С: 7.7. Бухгалтерия и в таблице MS Excel указывает категорию цены, которая зависит от периода.

% % При разработке сложного изделия с печатью менеджер формирует заявку дизайнеру на изготовление клише. Дизайнер разрабатывает макет клише в программе CorelDraw.  Готовый макет дизайнер высылает менеджеру в формате JPEG. 
% % Менеджер согласовывают с клиентом макет печати технологической карты  изделия.

% %Менеджер выставляет счёт покупателю на изготовление клише при необходимости. Дизайнер заказывает  изготовления печатной формы у стороннего производителя (см. процесс ''Учет оснастки \ref{bp:rigging}). 

% % При разработке нового изделия сложной высечки менеджер создает задание на разработку штанцформы дизайнеру на разработку и чертёж от клиента  в электронном виде. Форма заявки на разработку нового изделия пересылается менеджером по электронной почте или Skype дизайнеру Московский офис. 
% %Дизайнер в свою очередь передает задание конструкторское бюро конструктору для разработки штанцформы \ref{pic:pic_a36}.  Конструктор разрабатывает в программе AutoCAD  макет штанцевальной формы.  После согласования макет конструктор распечатывает  чертеж на большем принтер-плоттере в масштабе 1:1, прилагает чертёж на формате А4 и передает в отдел штампов.

% %Менеджер контролирует изготовление макета штанцевальной формы. 
% %Готовый макет конструктор высылает дизайнеру, дизайнер проверяет макет и высылает менеджеру (рис. \ref{pic:pic_d16.1}). 
% %Менеджер согласует конструкцию изделия с покупателем эскиз \ref{pic:pic_a3}. 
% %Стоимость штампа почти всегда за счёт заказчика.
% %При поступлении новой заявки с использованием штампа или/и клише, дизайнер, ориентируясь на технические параметры линий, распределяет на ту или иную перерабатывающую линию.

% % При необходимости печати на ящике менеджер создает заявку на изготовление печатной формы по форме \ref{pic:pic_d17}. Заявка с файлом дизайна от клиента высылается дизайнеру по электронной почте. Дизайнер размещает печать на штампе в программе Corel Draw или Adobe Illustrator.  Готовый макет дизайнер высылает менеджеру для согласования с клиентом. Менеджер согласует с клиентом технологическую карту (рис. \ref{pic:pic_d18}) макета печати на изделии. Стоимость изготовления печатной формы чаще всего включена в стоимость изделия. Заявки на изготовление штампа и печатной формы менеджер сохраняет в файлах в сетевой папке.
% % Дизайнер раскладывает эскиз на цвета, делает монтажный чертеж (рис. \ref{pic:pic_a7}).
% %Эскиз %согласовывают в увеличенном виде с каждой стороны короба форма \ref{pic:pic_a8}.


% %Дизайнер %%заказывает шаблоны для изготовления клише \ref{pic:pic_a7}. Дизайнер заносит информацию по клише в таблицу MS Excel в сетевой папке \ref{pic:pic_a2} и в локальную базу данных на Dephi.
% %Протяжки и буквы (сменность машинистов) также заказывает дизайнер, но их учет не ведется.

% % Дизайнер печатает технологическую карту в 3 экземплярах: одна остаётся в дизайнера, 1 передается начальнику участка флекс-форм, 1 передаётся в цех на линию. Дизайнер готовит чертежи для монтажа флекс-форм.

% %При изменении в дизайне в номер клише дизайнер прописывает цифру изменения (6514. 6 или 2), старая нумерация из базы удаляется. Само клише переклеивается фрагментами и всегда актуальное. При износе или повреждении клише из отдела монтажистов дизайнеру приходит служебная записка (рис. \ref{pic:pic_a11}) и заказывается новое.

% %Колорист разрабатывает рецептуры красок на новые заказы, делает выкраски на образцах и отдает менеджеру на согласование. На каждую рецептуру делается паспорт (рис. \ref{pic:photo66}), ведется реестр паспортов на краску (\ref{pic:photo67}). Колорист присутствует при выпуске первого заказа и изготавливает образец, который хранится в колировочной и выдается как эталон в дальнейшем на линии. 

% %Служба качества разрабатывает варианты упаковки по форме \ref{pic:pic_a47} и укладки на паллеты. %(44, 45,46). 
% %Бланки с вариантами упаковки хранятся на линиях. Служба качества распечатывает ярлыки на готовую продукцию (рис. \ref{pic:pic_a35}), на котором указывается вариант упаковки и укладки на паллет,  формат паллета.  Количество ярлыков на паллете определяется требованиями заказчика.

% % \begin{figure}
% % \begin{center}
% %   \includegraphics[height=0.94\textheight, width=0.94\textwidth, keepaspectratio]{Pics/a12.jpg}
% % \end{center}
% %   \caption{Дизайн печати на изделии}
% %   \label{pic:a12}
% % \end{figure}

%
\clearpage
\ifx \notincludehead\undefined
\normalsize
\end{document}
\fi