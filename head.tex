%\documentclass[12pt,russianb]{report}
%\usepackage[russian]{babel}
%\usepackage[cp1251]{inputenc}

\documentclass{report}
\usepackage{cmap}
\usepackage[T2A]{fontenc}

\usepackage[utf8]{inputenc}
\usepackage[english,russian]{babel}

\usepackage{longtable}   % подключение длинных таблиц
\usepackage[dvipsnames]{xcolor}
\usepackage{multirow}
\usepackage{array}
\usepackage{indentfirst} % идентификация первых абзацев после секционирования
\usepackage{lastpage}    % пакет достчета страниц 

\usepackage{fancyhdr}                    % расширенный формат страниц
\voffset=-25mm   % -25                   % сдвиг страницы вверх
\hoffset=-15mm   % -10     


  \usepackage[pdftex]{graphicx}            % загрузка графики под pdf
  \usepackage{cmap}                        % чтоб работал поиск по PDF 
  \usepackage[unicode, pdftex, colorlinks, linkcolor=blue]{hyperref}   % гиперссылки в PDF
  \pdfcompresslevel=9                      % сжимаем PDF 
  \textheight=240mm                        % для PDF высота печатного текста
  \textwidth=165mm                         % ширина печатного текста
  \renewcommand{\baselinestretch}{1.3}        % для PDF интервалы между
  \baselineskip=1.3\baselineskip              % строками 


\pagestyle{empty}
\pagestyle{fancy}
\lhead{\tiny ООО <<Опти-Софт>>}
\chead{}
\rhead{\tiny Отчет по обследованию производства \FIRMA}
\cfoot{\rule{\textwidth}{0.25pt}
~\arabic{page}}

\sloppy                             % подавление дополнительных переносов
\righthyphenmin=2                   % можно переносить
\setlength{\parindent}{10mm}        % отступ красной строки

\usepackage{todonotes}
%\newcommand{\todo}[1]{}
%\renewcommand{\todo}[1]{{\color{red} TODO: {#1}}}



\usepackage{placeins}    % пакет позволяет вставлять плавающие объекты (рисунки) в том месте, 
                         % где это необходимо. Для вывода рисунка после него встаить команду \FloatBarrier
                         
                         
                         

\newcommand{\red}[1]{\textcolor{Red}{#1}}
\newcommand{\green}[1]{\textcolor{Green}{#1}}
\newcommand{\blue}[1]{\textcolor{Blue}{#1}}                       