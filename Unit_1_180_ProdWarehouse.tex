\newpage
\subsection{Учет готовой продукции}
\label{bp:readygoods}

Приемка готовой продукции на склад кладовщиками не осуществляется. Документы о приемке на склад кладовщиками не оформляются. Готовая продукция считается переданной на склад, после того как машинист линии переработки в системе 1С: УПП произведет печать бирки со штрих-кодом на паллет. Учетчики ежедневно сверяют в системе 1С: УПП документы выработки производства и остатки на складе и при необходимости вносят корректировки. 

Водитель погрузчика перемещает паллеты на склад, но из-за нехватки площадей на складе готовая продукция часто остается в цехе. Адресное хранение готовой продукции отсутствует. Водители погрузчиков сообщают кладовщикам о местонахождении той или иной продукции.
%Съемщик на линии пишет рапорта (рис. \ref{pic:dd11}) и сдаточные акты (рис. \ref{pic:d17}) по изготовленной готовой продукции.
%Карщик привозит продукцию на склад со сдаточным актом от съемщика. На складе выделены примерные зоны размещения продукции по клиентам. Карщик знает зоны, где требуется разместить готовую продукцию на паллетах. Кладовщики не принимают участия в расстановке готовой продукции.

%На 2 этаж перемещается продукция только по выделенным клиентам. На предприятии выделено 4 зоны отгрузки: одна без пандуса, остальные с пандусом.

%Приемка продукции на склад выполняется кладовщиком по сдаточным актам от съемщика. 
%В течение дня кладовщик из сдаточного акта заносит в систему СБИС документ Выпуск изделий. Кладовщик в системе СБИС выбирает заявку клиента близкую по дате производства, которая по размерам и наименованию подходит к параметрам принимаемой продукции. Кладовщик в системе СБИС выбирает в заявке позицию и вводит принимаемое количество продукции, проводит (в обоработку). 
% Проводит часто. 
Поиск продукции для отгрузки выполняется кладовщиком только по наименованию продукции. 










